
% slide-5.tex

\documentclass[dvipdfmx,notheorems,t]{beamer}

\usepackage{docmute}

% settings.tex

\AtBeginSection[]{\frame[t]{\frametitle{目次}
  \tableofcontents[currentsection,hideallsubsections]}}

\AtBeginSubsection[]{\frame[t]{\frametitle{目次}
  \tableofcontents[currentsection,sectionstyle=show/hide,
  currentsubsection,subsectionstyle=show/shaded/hide]}}

\usefonttheme{professionalfonts}
\usetheme{Madrid}

\setbeamercovered{transparent=30} 
% \setbeamertemplate{navigation symbols}{}
\setbeamertemplate{frametitle}[default][left]
\setbeamertemplate{frametitle continuation}{}
\setbeamertemplate{enumerate items}[square]
\setbeamertemplate{caption}[numbered]

\let\oldframe\frame
\renewcommand\frame[1][t,allowdisplaybreaks,allowframebreaks]{\oldframe[#1]}

\addtobeamertemplate{block begin}{\setlength{\abovedisplayskip}{2.5pt}}

\usepackage{bxdpx-beamer}
\usepackage{pxjahyper}
\usepackage{minijs}

\usepackage{amsmath}
\usepackage{amssymb}
\usepackage{amsthm}
\usepackage{bm}
\usepackage{physics}

% Set the path to the figure
\graphicspath{{fig/}}

\usepackage{multirow}

% Add space in the table
\usepackage{cellspace}

% Add space in the table
\setlength\cellspacetoplimit{5pt}
\setlength\cellspacebottomlimit{5pt}

\usepackage{url}

% \hypersetup{
%   colorlinks = true,
%   urlcolor = blue,
%   linkcolor = black,
%   citecolor = green
% }

\DeclareMathOperator*{\argmax}{arg\,max}
\DeclareMathOperator*{\argmin}{arg\,min}
% \DeclareMathOperator{\Tr}{Tr}
% \DeclareMathOperator{\KL}{KL}
\DeclareMathOperator{\diag}{diag}
\DeclareMathOperator{\sgn}{sgn}
\DeclareMathOperator{\adj}{adj}
\DeclareMathOperator{\EOp}{\mathbb{E}}
\DeclareMathOperator{\HOp}{H}
\DeclareMathOperator{\KLOp}{KL}
\DeclareMathOperator{\VarOp}{Var}
\DeclareMathOperator{\CovOp}{Cov}
\newcommand\E[1]{\EOp \left[ #1 \right]}
\newcommand\Entropy[1]{\HOp \left[ #1 \right]}
\newcommand\MutualInfo[1]{I \left( #1 \right)}
\newcommand\KL[2]{\KLOp \left( #1 \parallel #2 \right)}
\newcommand\Var[1]{\VarOp \left[ #1 \right]}
\newcommand\Cov[2]{\CovOp \left( #1, #2 \right)}

\newcommand\BigO[1]{O \left( #1 \right)}
\newcommand\SmallO[1]{o \left( #1 \right)}

\newcommand\Comb[2]{{}_{#1}C_{#2}}

\newcommand{\middlerel}[1]{\mathrel{}\middle#1\mathrel{}}

\usepackage[T1]{fontenc}
\usepackage[utf8]{inputenc}

\setbeamertemplate{theorems}[numbered]
\theoremstyle{definition}
\newtheorem{theorem}{定理}
\newtheorem{definition}{定義}
\newtheorem{proposition}{命題}
\newtheorem{lemma}{補題}
\newtheorem{corollary}{系}
\newtheorem{conjecture}{予想}
\newtheorem*{remark}{Remark}
\renewcommand{\proofname}{}

\renewcommand{\figurename}{図}
\renewcommand{\tablename}{表}

\renewcommand{\kanjifamilydefault}{\gtdefault}



\title{行列輪講: 第5回 行列とベクトルの微分3}
\author{杉浦 圭祐}
\institute[松谷研究室]{慶應義塾大学理工学部情報工学科 松谷研究室}
\date{\today}

% Always use the \displaystyle
\everymath{\displaystyle}

\begin{document}

\linespread{1.1}

\frame{\titlepage}

\section{}

\begin{frame}[t,allowdisplaybreaks,allowframebreaks]{目次}
\tableofcontents
\end{frame}

\section{概要}

\begin{frame}{このスライドの概要}
\begin{itemize}
  \item 行列とベクトルの微分について確認する
  \begin{itemize}
    \item ヤコビの公式
    \item スカラの行列による微分 (行列式の入った微分)
    \item スカラのスカラによる微分 (ベクトルや行列を関数として含む場合)
  \end{itemize}
  \item 以下の資料も大変参考になります:
  \begin{itemize}
    \item \url{math.uwaterloo.ca/~hwolkowi/matrixcookbook.pdf}
    \item \url{comp.nus.edu.sg/cs5240/lecture/matrix-differentiation.pdf}
    \item \url{en.wikipedia.org/wiki/Matrix_calculus}
  \end{itemize}
\end{itemize}
\end{frame}

\section{ヤコビの公式}

\begin{frame}{余因子展開 (再掲)}
\begin{block}{余因子展開}
  $\vb{A}$を, $n$次正方行列とする. 各$i$行目と$j$列目について,
  \begin{align*}
    \det(\vb{A}) &= a_{i1} \Delta_{i1} + a_{i2} \Delta_{i2} + \cdots + a_{in} \Delta_{in}
      = \sum_j a_{ij} \Delta_{ij} \\
    \det(\vb{A}) &= a_{1j} \Delta_{1j} + a_{2j} \Delta_{2j} + \cdots + a_{nj} \Delta_{nj}
      = \sum_i a_{ij} \Delta_{ij}
  \end{align*}
\end{block}

\begin{itemize}
  \item 上式の$\Delta_{ij}$は, $\vb{A}$の\textcolor{red}{$(i, j)$余因子} (Cofactor) とよぶ.
  $$\Delta_{ij} = (-1)^{i + j} \det(\tilde{\vb{A}}_{ij})$$
  \item $\tilde{\vb{A}}_{ij}$は, $\vb{A}$から$i$行目と$j$列目を取り除いた, $n - 1$次行列である.
  \item $\Delta_{ij}$は, $i$行目と$j$行目の成分には依存しない.
\end{itemize}
\end{frame}

\begin{frame}{余因子展開 (再掲)}
\begin{block}{余因子展開}
  $\vb{A}$を, $n$次正方行列とする.
  $\Delta_{ij}$を$\vb{A}$の$(i, j)$余因子とする.
  \begin{align*}
    a_{i1} \Delta_{k1} + a_{i2} \Delta_{k2} + \cdots + a_{in} \Delta_{kn}
      &= \left\{ \begin{array}{ll} \det(\vb{A}) & i = k \\ 0 & \text{Otherwise} \end{array} \right. \\
    a_{1j} \Delta_{1k} + a_{2j} \Delta_{2k} + \cdots + a_{nj} \Delta_{nk}
      &= \left\{ \begin{array}{ll} \det(\vb{A}) & j = k \\ 0 & \text{Otherwise} \end{array} \right.
  \end{align*}
\end{block}
\end{frame}

\begin{frame}{余因子行列 (再掲)}
\begin{block}{余因子行列}
  $\vb{A}$を, $n$次行列とする.
  $\vb{A}$の$(i, j)$余因子$\Delta_{ij}$を並べた行列$\adj \vb{A}$を, $\vb{A}$の余因子行列という.
  $$\adj \vb{A} = \mqty(\Delta_{11} & \Delta_{21} & \cdots & \Delta_{n1} \\
    \Delta_{12} & \Delta_{22} & \cdots & \Delta_{n2} \\
    \vdots & \vdots & \ddots & \vdots \\
    \Delta_{1n} & \Delta_{2n} & \cdots & \Delta_{nn})$$
\end{block}

\begin{itemize}
  \item $\adj \vb{A}$の\textcolor{red}{$(i, j)$成分は, $(j, i)$余因子$\Delta_{ji}$}となる.
\end{itemize}
\end{frame}

\begin{frame}{余因子行列, 行列式, 逆行列 (再掲)}
\begin{block}{余因子行列, 行列式, 逆行列}
  $\vb{A}$の余因子行列$\adj \vb{A}$, 行列式$\det(\vb{A})$, 逆行列$\vb{A}^{-1}$について,
  $$\left( \adj \vb{A} \right) \vb{A} = \vb{A} \left( \adj \vb{A} \right)
    = \left( \det(\vb{A}) \right) \vb{I}$$
  $$\vb{A}^{-1} = \frac{1}{\det(\vb{A})} \adj \vb{A}$$
\end{block}
\end{frame}

\begin{frame}{ヤコビの公式 (Jacobi's Formula)}
\begin{block}{ヤコビの公式}
  \begin{align*}
    \pdv{\det(\vb{U})}{x} &= \tr(\adj \vb{U} \pdv{\vb{U}}{x})
      = \det(\vb{U}) \tr(\vb{U}^{-1} \pdv{\vb{U}}{x})
    & \text{($\vb{U} = \vb{U}(x)$)}
  \end{align*}
\end{block}

\begin{itemize}
  \item 行列式と, トレース, 逆行列が関連付けられる.
  \item 余因子行列, 行列式, 逆行列について, 以下が成り立つ ($\vb{U}$が正則であるとき).
  \begin{align*}
    \vb{U}^{-1} = \frac{1}{\det(\vb{U})} \adj \vb{U}
    \quad \longrightarrow \quad
    \adj \vb{U} = \det(\vb{U}) \vb{U}^{-1}
  \end{align*}
  \item Wikipediaに載っている証明を確認する.
  \begin{itemize}
    \item \url{en.wikipedia.org/wiki/Jacobi\%27s_formula}
  \end{itemize}
\end{itemize}
\end{frame}

\begin{frame}{ヤコビの公式 (Jacobi's Formula)}
最初に, 以下の補題を示す.
\begin{block}{補題}
  \begin{align*}
    \tr(\vb{A}^\top \vb{B}) = \sum_i \sum_j a_{ij} b_{ij}
  \end{align*}
\end{block}

左辺を順に計算すると, 次のようになる.
\begin{align*}
  \tr(\vb{A}^\top \vb{B}) &= \sum_j \left( \vb{A}^\top \vb{B} \right)_{jj} \\
    &= \sum_j \sum_i \left( \vb{A}^\top \right)_{ji} \left( \vb{B} \right)_{ij} \\
    &= \sum_j \sum_i a_{ij} b_{ij}
\end{align*}
\end{frame}

\begin{frame}{ヤコビの公式 (Jacobi's Formula)}
続いて, ヤコビの公式を示す.
\begin{block}{ヤコビの公式}
  \begin{align*}
    \pdv{\det(\vb{U})}{x} &= \tr(\adj \vb{U} \pdv{\vb{U}}{x})
    & \text{($\vb{U} = \vb{U}(x)$)}
  \end{align*}
\end{block}

$\vb{U}$の各成分を$u_{ij}$とする.
行列式$\det(\vb{U})$は, $\vb{U}$の全成分についての関数であるから,
合成関数の微分を使って, 次のようにかける.
\begin{align*}
  \pdv{\det(\vb{U})}{x} &= \sum_i \sum_j \pdv{\det(\vb{U})}{u_{ij}} \pdv{u_{ij}}{x} \\
    &= \sum_i \sum_j \pdv{\det(\vb{U})}{u_{ij}} \left( \pdv{\vb{U}}{x} \right)_{ij}
\end{align*}
次に, $\pdv{\det(\vb{U})}{u_{ij}}$を, $\vb{U}$の余因子展開によって求める.
\end{frame}

\begin{frame}{ヤコビの公式 (Jacobi's Formula)}
$\vb{U}$の余因子展開は, 次のようになって, 全ての$k$について成立する.
\begin{align*}
  \det(\vb{U}) = \sum_l u_{kl} \Delta_{kl}
\end{align*}
$k$は自由に選べるが, ここでは$k = i$として, $u_{ij}$により微分すると,
\begin{align*}
  \pdv{\det(\vb{U})}{u_{ij}} &= \pdv{u_{ij}} \sum_l u_{il} \Delta_{il}
    = \sum_l \left( \Delta_{il} \pdv{u_{il}}{u_{ij}} + u_{il} \pdv{\Delta_{il}}{u_{ij}} \right)
\end{align*}
$\Delta_{ij} = (-1)^{i + j} \det(\tilde{\vb{U}}_{ij})$である.
$\tilde{\vb{U}}_{ij}$は, $\vb{U}$から$i$行目と$j$列目を取り除いた行列であるから,
\textcolor{red}{$u_{ij}$には依存しない定数項}.
よって, $\pdv{\Delta_{il}}{u_{ij}} = 0$. \\
また, $\pdv{u_{il}}{u_{ij}}$は, $l = j$のときのみ$1$となるから, $\pdv{u_{il}}{u_{ij}} = \delta_{lj}$.
\end{frame}

\begin{frame}{ヤコビの公式 (Jacobi's Formula)}
$\pdv{\Delta_{il}}{u_{ij}} = 0$と$\pdv{u_{il}}{u_{ij}} = \delta_{lj}$を代入すれば,
\begin{align*}
  \pdv{\det(\vb{U})}{u_{ij}}
    &= \sum_l \left( \Delta_{il} \pdv{u_{il}}{u_{ij}} + u_{il} \pdv{\Delta_{il}}{u_{ij}} \right)
    = \sum_l \left( \Delta_{il} \delta_{lj} \right) = \Delta_{ij}
\end{align*}
$\adj \vb{U}$の$(j, i)$成分は, $\vb{U}$の$(i, j)$余因子$\Delta_{ij}$であるから,
\begin{align*}
  \pdv{\det(\vb{U})}{u_{ij}} = \Delta_{ij}
    = \left( \adj \vb{U} \right)_{ji} = \left( \adj^\top \vb{U} \right)_{ij}
\end{align*}
これを代入して,
\begin{align*}
  \pdv{\det(\vb{U})}{x}
    &= \sum_i \sum_j \pdv{\det(\vb{U})}{u_{ij}} \left( \pdv{\vb{U}}{x} \right)_{ij}
    = \sum_i \sum_j \left( \adj^\top \vb{U} \right)_{ij} \left( \pdv{\vb{U}}{x} \right)_{ij}
\end{align*}
補題$\tr(\vb{A}^\top \vb{B}) = \sum_i \sum_j a_{ij} b_{ij}$より, ヤコビの公式が得られる:
\begin{align*}
  \pdv{\det(\vb{U})}{x}
    = \sum_i \sum_j \left( \adj^\top \vb{U} \right)_{ij} \left( \pdv{\vb{U}}{x} \right)_{ij}
    = \tr(\adj \vb{U} \pdv{\vb{U}}{x})
\end{align*}
$\vb{U}^{-1} = \frac{1}{\det(\vb{U})} \adj \vb{U}$を代入すれば,
\begin{align*}
  \pdv{\det(\vb{U})}{x} = \det(\vb{U}) \tr(\vb{U}^{-1} \pdv{\vb{U}}{x})
\end{align*}
\end{frame}

\section{スカラの行列による微分}
\subsection{行列式を含む微分}

\begin{frame}{スカラの行列による微分}
\begin{block}{ヤコビの公式, 行列式の対数}
  \begin{align*}
    \pdv{\det(\vb{U})}{x} &= \tr(\adj \vb{U} \pdv{\vb{U}}{x})
      = \det(\vb{U}) \tr(\vb{U}^{-1} \pdv{\vb{U}}{x})
      & \text{($\vb{U} = \vb{U}(x)$)} \\
    \pdv{\ln(\det(\vb{U}))}{x} &= \tr(\vb{U}^{-1} \pdv{\vb{U}}{x})
      & \text{($\vb{U} = \vb{U}(x)$)}
  \end{align*}
\end{block}

2つ目の式は, 以下のように示せる.
$y = \det(\vb{U})$とおくと,
\begin{align*}
  \pdv{\ln(\det(\vb{U}))}{x} &= \pdv{\ln y}{x} = \pdv{\ln y}{y} \pdv{y}{x} = \frac{1}{y} \pdv{y}{x}
    = \frac{1}{\det(\vb{U})} \pdv{\det(\vb{U})}{x} \\
    &= \frac{1}{\det(\vb{U})} \cdot \det(\vb{U}) \tr(\vb{U}^{-1} \pdv{\vb{U}}{x})
    = \tr(\vb{U}^{-1} \pdv{\vb{U}}{x})
\end{align*}
\end{frame}

\begin{frame}{スカラの行列による微分}
\begin{block}{行列式の2次微分}
  \begin{align*}
    \pdv[2]{\det(\vb{U})}{x} &= \det(\vb{U}) \bigg\{
      \tr(\vb{U}^{-1} \pdv[2]{\vb{U}}{x}) \\
      & \quad + \tr^2 \left( \vb{U}^{-1} \pdv{\vb{U}}{x} \right)
      - \tr(\left( \vb{U}^{-1} \pdv{\vb{U}}{x} \right)^2) \bigg\}
      \quad \quad \text{($\vb{U} = \vb{U}(x)$)}
  \end{align*}
\end{block}

以下のように示せる.
合成関数の微分より,
\begin{align*}
  \pdv[2]{\det(\vb{U})}{x} &= \pdv{x} \det(\vb{U}) \tr(\vb{U}^{-1} \pdv{\vb{U}}{x}) \\
    &= \pdv{\det(\vb{U})}{x} \tr(\vb{U}^{-1} \pdv{\vb{U}}{x})
      + \det(\vb{U}) \pdv{\tr(\vb{U}^{-1} \pdv{\vb{U}}{x})}{x}
\end{align*}

$\pdv{\tr(\vb{U} \vb{V})}{x} = \tr(\pdv{\vb{U}}{x} \vb{V}) + \tr(\vb{U} \pdv{\vb{V}}{x})$より,
\begin{align*}
    \pdv[2]{\det(\vb{U})}{x}
    &= \pdv{\det(\vb{U})}{x} \tr(\vb{U}^{-1} \pdv{\vb{U}}{x})
      + \det(\vb{U}) \pdv{\tr(\vb{U}^{-1} \pdv{\vb{U}}{x})}{x} \\
    &= \det(\vb{U}) \tr(\vb{U}^{-1} \pdv{\vb{U}}{x}) \tr(\vb{U}^{-1} \pdv{\vb{U}}{x}) \\
      & \quad + \det(\vb{U}) \left( \tr(\pdv{\vb{U}^{-1}}{x} \pdv{\vb{U}}{x})
      + \tr(\vb{U}^{-1} \pdv[2]{\vb{U}}{x}) \right)
\end{align*}
\newpage

式変形を続けると, 次のようになる.
\begin{align*}
  \pdv[2]{\det(\vb{U})}{x}
    &= \det(\vb{U}) \tr(\vb{U}^{-1} \pdv{\vb{U}}{x}) \tr(\vb{U}^{-1} \pdv{\vb{U}}{x}) \\
      & \quad + \det(\vb{U}) \left( \tr(\pdv{\vb{U}^{-1}}{x} \pdv{\vb{U}}{x})
      + \tr(\vb{U}^{-1} \pdv[2]{\vb{U}}{x}) \right) \\
    &= \det(\vb{U}) \tr(\vb{U}^{-1} \pdv{\vb{U}}{x}) \tr(\vb{U}^{-1} \pdv{\vb{U}}{x}) \\
      & \quad + \det(\vb{U}) \tr(-\vb{U}^{-1} \pdv{\vb{U}}{x} \vb{U}^{-1} \pdv{\vb{U}}{x})
      + \det(\vb{U}) \tr(\vb{U}^{-1} \pdv[2]{\vb{U}}{x}) \\
    &= \det(\vb{U}) \bigg\{ \tr(\vb{U}^{-1} \pdv[2]{\vb{U}}{x})
      + \tr^2 \left(\vb{U}^{-1} \pdv{\vb{U}}{x} \right) \\
      & \quad - \tr(\left( \vb{U}^{-1} \pdv{\vb{U}}{x} \right)^2) \bigg\}
\end{align*}
\end{frame}

\begin{frame}{スカラの行列による微分}
\begin{block}{行列式}
  \begin{align*}
    \pdv{\det(\vb{X})}{\vb{X}} &= \adj \vb{X} = \det(\vb{X}) \vb{X}^{-1} \\
    \pdv{\det(a \vb{X})}{\vb{X}}
      &= a \adj \left( a \vb{X} \right) = \det(a \vb{X}) \vb{X}^{-1} & \text{($a$は定数)}
  \end{align*}
\end{block}

以下のように, 要素ごとに確認できる.
余因子展開$\det(\vb{X}) = \sum_l x_{lk} \Delta_{lk}$について, $k = i$を選べば,
\begin{align*}
  \left( \pdv{\det(\vb{X})}{\vb{X}} \right)_{ij}
    &= \pdv{\det(\vb{X})}{x_{ji}}
    = \pdv{x_{ji}} \sum_l x_{li} \Delta_{li} \\
    &= \sum_l \left( \Delta_{li} \pdv{x_{li}}{x_{ji}} + x_{li} \pdv{\Delta_{li}}{x_{ji}} \right)
\end{align*}
$\vb{X}$の$(l, i)$余因子$\Delta_{li} = (-1)^{l + i} \det(\tilde{\vb{X}}_{li})$について,
$\tilde{\vb{X}}_{li}$は, $\vb{X}$から$l$行目と$i$列目を除いた行列であるから,
$\Delta_{li}$は$x_{ji}$には依存しない定数項.
よって, $\pdv{\Delta_{li}}{x_{ji}} = 0$.
また, $\pdv{x_{li}}{x_{ji}} = \delta_{lj}$.
よって,
\begin{align*}
  \left( \pdv{\det(\vb{X})}{\vb{X}} \right)_{ij}
    &= \sum_l \left( \Delta_{li} \pdv{x_{li}}{x_{ji}} + x_{li} \pdv{\Delta_{li}}{x_{ji}} \right)
    = \sum_l \Delta_{li} \delta_{lj} = \Delta_{ji}
\end{align*}
余因子行列$\adj \vb{X}$の$(i, j)$成分は, $\vb{X}$の$(j, i)$余因子$\Delta_{ji}$であるから,
\begin{align*}
  \left( \pdv{\det(\vb{X})}{\vb{X}} \right)_{ij} = \Delta_{ji} = \left( \adj \vb{X} \right)_{ij}
\end{align*}
$\pdv{\det(a \vb{X})}{\vb{X}}$についても, 同様の流れで確認できる.
\end{frame}

\begin{frame}{スカラの行列による微分}
\begin{block}{行列式の対数 (重要な式の1つ)}
  \begin{align*}
    \pdv{\ln(\det(a \vb{X}))}{\vb{X}} &= \vb{X}^{-1} & \text{($a$は定数)} \\
    \pdv{\ln(\det(\vb{X}))}{\vb{X}} &= \vb{X}^{-1}
  \end{align*}
\end{block}

以下のように, 要素ごとに確認できる.
合成関数の微分より,
\begin{align*}
  \left( \pdv{\ln(\det(a \vb{X}))}{\vb{X}} \right)_{ij}
    &= \pdv{\ln(\det(a \vb{X}))}{x_{ji}}
    = \frac{1}{\det(a \vb{X})} \pdv{\det(a \vb{X})}{x_{ji}} \\
    &= \frac{1}{\det(a \vb{X})} \left( \pdv{\det(a \vb{X})}{\vb{X}} \right)_{ij} \\
    &= \frac{1}{\det(a \vb{X})} \left( \det(a \vb{X}) \vb{X}^{-1} \right)_{ij}
    = \left( \vb{X}^{-1} \right)_{ij}
\end{align*}
\end{frame}

\begin{frame}{スカラの行列による微分}
\begin{block}{行列積の行列式}
  $\vb{A}, \vb{B}$が正方行列であるとき,
  \begin{align*}
    \pdv{\det(\vb{A} \vb{X} \vb{B})}{\vb{X}} &= \det(\vb{A} \vb{X} \vb{B}) \vb{X}^{-1}
      & \text{($\vb{A}, \vb{B}$は定数)}
  \end{align*}
  $\vb{A}, \vb{B}$が正方行列ではないとき,
  \begin{align*}
    \pdv{\det(\vb{A} \vb{X} \vb{B})}{\vb{X}}
      &= \det(\vb{A} \vb{X} \vb{B}) \vb{B} \left( \vb{A} \vb{X} \vb{B} \right)^{-1} \vb{A}
      & \text{($\vb{A}, \vb{B}$は定数)}
  \end{align*}
\end{block}

$\vb{A}, \vb{B}$が正方行列であるとき,
$\det(\vb{A} \vb{B} \vb{C}) = \det(\vb{A}) \det(\vb{B}) \det(\vb{C})$を用いて示せる (練習問題).
% \begin{align*}
%   \pdv{\det(\vb{A} \vb{X} \vb{B})}{\vb{X}} &= \det(\vb{A}) \det(\vb{B}) \pdv{\det(\vb{X})}{\vb{X}} \\
%     &= \det(\vb{A}) \det(\vb{B}) \det(\vb{X}) \vb{X}^{-1} \\
%     &= \det(\vb{A} \vb{X} \vb{B}) \vb{X}^{-1}
% \end{align*}
\newpage

$\vb{A}, \vb{B}$が正方行列ではないとき, 以下のように, 要素ごとに確認できる.
$\vb{U} = \vb{A} \vb{X} \vb{B}$とおき, ヤコビの公式を用いる.
\begin{align*}
  \left( \pdv{\det(\vb{A} \vb{X} \vb{B})}{\vb{X}} \right)_{ij}
    &= \pdv{\det(\vb{U})}{x_{ji}}
    = \det(\vb{U}) \tr(\vb{U}^{-1} \pdv{\vb{U}}{x_{ji}})
\end{align*}
ここで, 行列積のスカラによる微分から,
\begin{align*}
  \pdv{\vb{U}}{x_{ji}} &= \pdv{\vb{A} \vb{X} \vb{B}}{x_{ji}}
    = \vb{A} \pdv{\vb{X}}{x_{ji}} \vb{B}
    = \vb{A} \vb{J}^{ji} \vb{B} \\
    & \text{($\vb{J}^{ji}$は, $(j, i)$成分のみが$1$で, それ以外が$0$であるような行列)}
\end{align*}
トレースを計算すると,
\begin{align*}
  \tr(\vb{U}^{-1} \pdv{\vb{U}}{x_{ji}}) &= \tr(\vb{U}^{-1} \vb{A} \vb{J}^{ji} \vb{B})
    = \sum_k \left( \vb{U}^{-1} \vb{A} \vb{J}^{ji} \vb{B} \right)_{kk}
\end{align*}
\newpage

$\vb{U}^{-1}$の各成分を$z_{ij}$とおき, 式変形を続けると,
\begin{align*}
  & \tr(\vb{U}^{-1} \pdv{\vb{U}}{x_{ji}})
    = \sum_k \left( \vb{U}^{-1} \vb{A} \vb{J}^{ji} \vb{B} \right)_{kk} \\
    &= \sum_k \sum_l \sum_m \sum_n z_{kl} a_{lm} \left( \vb{J}^{ji} \right)_{mn} b_{nk}
    = \sum_k \sum_l \sum_m \sum_n z_{kl} a_{lm} \delta_{jm} \delta_{in} b_{nk} \\
    &= \sum_k \sum_l z_{kl} a_{lj} b_{ik}
    = \sum_k b_{ik} \left( \vb{U}^{-1} \vb{A} \right)_{kj}
    = \left( \vb{B} \vb{U}^{-1} \vb{A} \right)_{ij}
\end{align*}
ただし, $\left( \vb{J}^{ji} \right)_{mn} = \delta_{jm} \delta_{in}$.
よって ($\vb{U} = \vb{A} \vb{X} \vb{B}$),
\begin{align*}
  \left( \pdv{\det(\vb{A} \vb{X} \vb{B})}{\vb{X}} \right)_{ij}
    &= \det(\vb{U}) \tr(\vb{U}^{-1} \pdv{\vb{U}}{x_{ji}})
    = \det(\vb{U}) \left( \vb{B} \vb{U}^{-1} \vb{A} \right)_{ij} \\
    &= \det(\vb{A} \vb{X} \vb{B})
      \left( \vb{B} \left( \vb{A} \vb{X} \vb{B} \right)^{-1} \vb{A} \right)_{ij}
\end{align*}
\end{frame}

\begin{frame}{スカラの行列による微分}
\begin{block}{累乗の行列式}
  \begin{align*}
    \pdv{\det(\vb{X}^n)}{\vb{X}} &= n \det(\vb{X}^n) \vb{X}^{-1}
  \end{align*}
\end{block}

以下のように, 要素ごとに確認できる.
$\det(\vb{A}^n) = \left( \det(\vb{A}) \right)^n$と, 合成関数の微分より,
\begin{align*}
  & \left( \pdv{\det(\vb{X}^n)}{\vb{X}} \right)_{ij}
    = \pdv{\det(\vb{X}^n)}{x_{ji}}
    = \pdv{x_{ji}} \left( \det(\vb{X}) \right)^n \\
    &= n \left( \det(\vb{X}) \right)^{n - 1} \pdv{\det(\vb{X})}{x_{ji}}
    = n \left( \det(\vb{X}) \right)^{n - 1} \left( \pdv{\det(\vb{X})}{\vb{X}} \right)_{ij} \\
    &= n \left( \det(\vb{X}) \right)^{n - 1} \det(\vb{X}) \left( \vb{X}^{-1} \right)_{ij}
    = n \left( \det(\vb{X}) \right)^n \left( \vb{X}^{-1} \right)_{ij} \\
    &= n \det(\vb{X}^n) \left( \vb{X}^{-1} \right)_{ij}
\end{align*}
\end{frame}

\begin{frame}{スカラの行列による微分}
\begin{block}{二次式の行列式}
  $\vb{X}$が正方行列で, 正則とすると,
  \begin{align*}
    \pdv{\det(\vb{X}^\top \vb{A} \vb{X})}{\vb{X}} &= 2 \det(\vb{X}^\top \vb{A} \vb{X}) \vb{X}^{-1}
      & \text{($\vb{A}$は定数)}
  \end{align*}
\end{block}

以下のように示せる.
$\det(\vb{A}^\top) = \det(\vb{A})$を用いる.
\begin{align*}
  \pdv{\det(\vb{X}^\top \vb{A} \vb{X})}{\vb{X}}
    &= \pdv{\vb{X}} \det(\vb{A}) \det(\vb{X}) \det(\vb{X}^\top)
    = \det(\vb{A}) \pdv{\vb{X}} \left( \det(\vb{X}) \right)^2 \\
    &= \det(\vb{A}) \pdv{\vb{X}} \det(\vb{X}^2)
    = 2 \det(\vb{A}) \det(\vb{X}^2) \vb{X}^{-1} \\
    &= 2 \det(\vb{A}) \det(\vb{X}) \det(\vb{X}) \vb{X}^{-1} \\
    &= 2 \det(\vb{X}) \det(\vb{A}) \det(\vb{X}^\top) \vb{X}^{-1}
    = 2 \det(\vb{X}^\top \vb{A} \vb{X}) \vb{X}^{-1}
\end{align*}
\end{frame}

\begin{frame}{スカラの行列による微分}
\begin{block}{二次式の行列式}
  $\vb{X}$が正方行列ではないとき,
  \begin{align*}
    \pdv{\det(\vb{X}^\top \vb{A} \vb{X})}{\vb{X}}
      &= \det(\vb{X}^\top \vb{A} \vb{X}) \bigg(
        \left( \vb{X}^\top \vb{A}^\top \vb{X} \right)^{-1} \vb{X}^\top \vb{A}^\top \\
      & \quad \quad + \left( \vb{X}^\top \vb{A} \vb{X} \right)^{-1} \vb{X}^\top \vb{A} \bigg)
      & \text{($\vb{A}$は定数)}
  \end{align*}
\end{block}

以下のように, 要素ごとに確認できる.
$\vb{U} = \vb{X}^\top \vb{A} \vb{X}$とおき, ヤコビの公式を用いる.
\begin{align*}
  \left( \pdv{\det(\vb{X}^\top \vb{A} \vb{X})}{\vb{X}} \right)_{ij}
    &= \pdv{\det(\vb{U})}{x_{ji}}
    = \det(\vb{U}) \tr(\vb{U}^{-1} \pdv{\vb{U}}{x_{ji}})
\end{align*}
\newpage

ここで, 行列積のスカラによる微分から,
\begin{align*}
  \pdv{\vb{U}}{x_{ji}} &= \pdv{\vb{X}^\top \vb{A} \vb{X}}{x_{ji}}
    = \pdv{\vb{X}^\top}{x_{ji}} \vb{A} \vb{X} + \vb{X}^\top \pdv{\vb{A}}{x_{ji}} \vb{X}
      + \vb{X}^\top \vb{A} \pdv{\vb{X}}{x_{ji}} \\
    &= \left( \vb{J}^{ji} \right)^\top \vb{A} \vb{X} + \vb{X}^\top \vb{A} \vb{J}^{ji} \\
    & \text{($\vb{J}^{ji}$は, $(j, i)$成分のみが$1$で, それ以外が$0$であるような行列)}
\end{align*}
トレースを計算すると,
\begin{align*}
  \tr(\vb{U}^{-1} \pdv{\vb{U}}{x_{ji}})
    &= \tr(\vb{U}^{-1} \left(
      \left( \vb{J}^{ji} \right)^\top \vb{A} \vb{X}
      + \vb{X}^\top \vb{A} \vb{J}^{ji} \right)) \\
    &= \sum_k \left( \vb{U}^{-1} \left(
      \left( \vb{J}^{ji} \right)^\top \vb{A} \vb{X}
      + \vb{X}^\top \vb{A} \vb{J}^{ji} \right) \right)_{kk}
\end{align*}
\newpage

$\vb{U}^{-1}$の各成分を$z_{ij}$とする.
$\left( \vb{J}^{ji} \right)_{lm} = \delta_{jl} \delta_{im}$であるから,
\begin{align*}
  & \tr(\vb{U}^{-1} \pdv{\vb{U}}{x_{ji}})
    = \sum_k \left( \vb{U}^{-1} \left(
      \left( \vb{J}^{ji} \right)^\top \vb{A} \vb{X}
      + \vb{X}^\top \vb{A} \vb{J}^{ji} \right) \right)_{kk} \\
    &= \sum_k \sum_l z_{kl} \left(
      \left( \vb{J}^{ji} \right)^\top \vb{A} \vb{X}
      + \vb{X}^\top \vb{A} \vb{J}^{ji} \right)_{lk} \\
    &= \sum_k \sum_l z_{kl} \left(
      \sum_m \sum_n \left( \vb{J}^{ji} \right)_{ml} a_{mn} x_{nk}
      + \sum_m \sum_n x_{ml} a_{mn} \left( \vb{J}^{ji} \right)_{nk} \right) \\
    &= \sum_k \sum_l z_{kl} \left(
      \sum_m \sum_n \delta_{jm} \delta_{il} a_{mn} x_{nk}
      + \sum_m \sum_n x_{ml} a_{mn} \delta_{jn} \delta_{ik} \right) \\
    &= \sum_k z_{ki} \sum_n a_{jn} x_{nk} + \sum_l z_{il} \sum_m x_{ml} a_{mj}
\end{align*}
\newpage

整理すると,
\begin{align*}
  & \tr(\vb{U}^{-1} \pdv{\vb{U}}{x_{ji}})
    = \sum_k z_{ki} \sum_n a_{jn} x_{nk} + \sum_l z_{il} \sum_m x_{ml} a_{mj} \\
  &= \sum_k z_{ki} \left( \vb{X}^\top \vb{A}^\top \right)_{kj}
    + \sum_l z_{il} \left( \vb{X}^\top \vb{A} \right)_{lj} \\
  &= \left( \left( \vb{U}^{-1} \right)^\top \vb{X}^\top \vb{A}^\top \right)_{ij}
    + \left( \vb{U}^{-1} \vb{X}^\top \vb{A} \right)_{ij}
\end{align*}
これを, 以下の式に代入し直す ($\vb{U} = \vb{X}^\top \vb{A} \vb{X}$).
\begin{align*}
  & \left( \pdv{\det(\vb{X}^\top \vb{A} \vb{X})}{\vb{X}} \right)_{ij}
    = \pdv{\det(\vb{U})}{x_{ji}}
    = \det(\vb{U}) \tr(\vb{U}^{-1} \pdv{\vb{U}}{x_{ji}})
\end{align*}
\newpage

以上をまとめると,
\begin{align*}
  & \left( \pdv{\det(\vb{X}^\top \vb{A} \vb{X})}{\vb{X}} \right)_{ij}
    = \pdv{\det(\vb{U})}{x_{ji}}
    = \det(\vb{U}) \tr(\vb{U}^{-1} \pdv{\vb{U}}{x_{ji}}) \\
    &= \det(\vb{U}) \left( \left( \vb{U}^{-1} \right)^\top \vb{X}^\top \vb{A}^\top
      + \vb{U}^{-1} \vb{X}^\top \vb{A} \right)_{ij} \\
    &= \det(\vb{X}^\top \vb{A} \vb{X}) \left(
      \left( \vb{X}^\top \vb{A}^\top \vb{X} \right)^{-1} \vb{X}^\top \vb{A}^\top
      + \left( \vb{X}^\top \vb{A} \vb{X} \right)^{-1} \vb{X}^\top \vb{A} \right)_{ij}
\end{align*}

ただし, $\left( \vb{A}^{-1} \right)^\top = \left( \vb{A}^\top \right)^{-1}$と,
$\left( \vb{A} \vb{B} \vb{C} \right)^\top = \vb{C}^\top \vb{B}^\top \vb{A}^\top$であるから,
\begin{align*}
  \left( \vb{U}^{-1} \right)^\top &= \left( \vb{U}^\top \right)^{-1}
    = \left( \left( \vb{X}^\top \vb{A} \vb{X} \right)^\top \right)^{-1}
    = \left( \vb{X}^\top \vb{A}^\top \vb{X} \right)^{-1}
\end{align*}
\end{frame}

\begin{frame}{スカラの行列による微分}
\begin{block}{二次式の行列式}
  $\vb{X}$が正方行列ではなく, $\vb{A}$が対称行列であるとき,
  \begin{align*}
    \pdv{\det(\vb{X}^\top \vb{A} \vb{X})}{\vb{X}}
      &= 2 \det(\vb{X}^\top \vb{A} \vb{X})
        \left( \vb{X}^\top \vb{A} \vb{X} \right)^{-1} \vb{X}^\top \vb{A} \\
    \pdv{\det(\vb{X}^\top \vb{X})}{\vb{X}}
      &= 2 \det(\vb{X}^\top \vb{X}) \left( \vb{X}^\top \vb{X} \right)^{-1} \vb{X}^\top
      = 2 \det(\vb{X}^\top \vb{X}) \vb{X}^\dagger \\
      & \text{($\vb{A}$は定数, $\vb{X}^\dagger = \left( \vb{X}^\top \vb{X} \right)^{-1} \vb{X}^\top$は$\vb{X}$の擬似逆行列)}
  \end{align*}
\end{block}

以下の式に, $\vb{A} = \vb{A}^\top$を代入して整理すればよい (練習問題).
\begin{align*}
  \pdv{\det(\vb{X}^\top \vb{A} \vb{X})}{\vb{X}}
    &= \det(\vb{X}^\top \vb{A} \vb{X}) \bigg(
      \left( \vb{X}^\top \vb{A}^\top \vb{X} \right)^{-1} \vb{X}^\top \vb{A}^\top \\
    & \quad \quad + \left( \vb{X}^\top \vb{A} \vb{X} \right)^{-1} \vb{X}^\top \vb{A} \bigg)
\end{align*}
\end{frame}

\begin{frame}{スカラの行列による微分}
\begin{block}{二次式の行列式の対数}
  \begin{align*}
    \pdv{\ln(\det(\vb{X}^\top \vb{X}))}{\vb{X}} &= 2 \vb{X}^\dagger
      & \text{($\vb{X}^\dagger = \left( \vb{X}^\top \vb{X} \right)^{-1} \vb{X}^\top$は, $\vb{X}$の擬似逆行列)}
  \end{align*}
\end{block}

以下のように, 要素ごとに確認できる.
\begin{align*}
  & \left( \pdv{\ln(\det(\vb{X}^\top \vb{X}))}{\vb{X}} \right)_{ij}
    = \pdv{\ln(\det(\vb{X}^\top \vb{X}))}{x_{ji}}
    = \frac{1}{\det(\vb{X}^\top \vb{X})} \pdv{\det(\vb{X}^\top \vb{X})}{x_{ji}} \\
    &= \frac{1}{\det(\vb{X}^\top \vb{X})} \left( \pdv{\det(\vb{X}^\top \vb{X})}{\vb{X}} \right)_{ij}
    = \frac{1}{\det(\vb{X}^\top \vb{X})} \left( 2 \det(\vb{X}^\top \vb{X}) \vb{X}^\dagger \right)_{ij} \\
    &= 2 \left( \vb{X}^\dagger \right)_{ij}
\end{align*}
\end{frame}

\begin{frame}{スカラの行列による微分}
\begin{block}{二次式の行列式の対数}
  \begin{align*}
    \pdv{\ln(\det(\vb{X}^\top \vb{X}))}{\vb{X}^\dagger} &= -2 \vb{X}
      & \text{($\vb{X}^\dagger = \left( \vb{X}^\top \vb{X} \right)^{-1} \vb{X}^\top$は, $\vb{X}$の擬似逆行列)}
  \end{align*}
\end{block}

$\vb{X}^\dagger$の各成分を$z_{ij}$とする.
以下のように, 要素ごとに調べる.
\begin{align*}
  & \left( \pdv{\ln(\det(\vb{X}^\top \vb{X}))}{\vb{X}^\dagger} \right)_{ij}
    = \pdv{\ln(\det(\vb{X}^\top \vb{X}))}{z_{ji}}
    = \frac{1}{\det(\vb{X}^\top \vb{X})} \pdv{\det(\vb{X}^\top \vb{X})}{z_{ji}} \\
    &= \frac{1}{\det(\vb{X}^\top \vb{X})} \det(\vb{X}^\top \vb{X})
      \tr(\left( \vb{X}^\top \vb{X} \right)^{-1} \pdv{\vb{X}^\top \vb{X}}{z_{ji}}) \\
    &= \tr(\left( \vb{X}^\top \vb{X} \right)^{-1}
      \left( \vb{X}^\top \pdv{\vb{X}}{z_{ji}} + \pdv{\vb{X}^\top}{z_{ji}} \vb{X} \right)) \\
\end{align*}
\newpage

$\tr(\vb{A}^\top) = \tr(\vb{A})$,
$\left( \vb{A}^{-1} \right)^\top = \left( \vb{A}^\top \right)^{-1}$と, トレースの循環性より,
\begin{align*}
  & \left( \pdv{\ln(\det(\vb{X}^\top \vb{X}))}{\vb{X}^\dagger} \right)_{ij}
    = \tr(\left( \vb{X}^\top \vb{X} \right)^{-1}
      \left( \vb{X}^\top \pdv{\vb{X}}{z_{ji}} + \pdv{\vb{X}^\top}{z_{ji}} \vb{X} \right)) \\
    &= \tr(\left( \vb{X}^\top \vb{X} \right)^{-1} \vb{X}^\top \pdv{\vb{X}}{z_{ji}})
      + \tr(\left( \vb{X}^\top \vb{X} \right)^{-1} \pdv{\vb{X}^\top}{z_{ji}} \vb{X}) \\
    &= \tr(\left( \vb{X}^\top \vb{X} \right)^{-1} \vb{X}^\top \pdv{\vb{X}}{z_{ji}})
      + \tr(\vb{X}^\top \pdv{\vb{X}}{z_{ji}} \left( \vb{X}^\top \vb{X} \right)^{-1})
      \quad (\because \text{転置}) \\
    &= \tr(\left( \vb{X}^\top \vb{X} \right)^{-1} \vb{X}^\top \pdv{\vb{X}}{z_{ji}})
      + \tr(\left( \vb{X}^\top \vb{X} \right)^{-1} \vb{X}^\top \pdv{\vb{X}}{z_{ji}})
      \quad (\because \text{循環性}) \\
    &= 2 \tr(\left( \vb{X}^\top \vb{X} \right)^{-1} \vb{X}^\top \pdv{\vb{X}}{z_{ji}})
\end{align*}
\newpage

$\pdv{\vb{X}}{z_{ji}} = \left( \pdv{\vb{X}}{\vb{X}^\dagger} \right)_{ij}$を調べる.
$\vb{X} = \vb{Y}^\dagger, \vb{Y} = \vb{X}^\dagger$と置き換えると,
\begin{align*}
  \left( \pdv{\vb{X}}{\vb{X}^\dagger} \right)_{ij}
    &= \left( \pdv{\vb{Y}^\dagger}{\vb{Y}} \right)_{ij}
    = \left( \pdv{\left( \vb{Y}^\top \vb{Y} \right)^{-1} \vb{Y}^\top}{\vb{Y}} \right)_{ij}
    = \pdv{\left( \vb{Y}^\top \vb{Y} \right)^{-1} \vb{Y}^\top}{y_{ji}}
\end{align*}

行列積のスカラによる微分, 逆行列の微分から,
\begin{align*}
  & \pdv{\left( \vb{Y}^\top \vb{Y} \right)^{-1} \vb{Y}^\top}{y_{ji}}
    = \pdv{\left( \vb{Y}^\top \vb{Y} \right)^{-1}}{y_{ji}} \vb{Y}^\top
      + \left( \vb{Y}^\top \vb{Y} \right)^{-1} \pdv{\vb{Y}^\top}{y_{ji}} \\
    &= - \left( \vb{Y}^\top \vb{Y} \right)^{-1} \pdv{\vb{Y}^\top \vb{Y}}{y_{ji}}
      \left( \vb{Y}^\top \vb{Y} \right)^{-1} \vb{Y}^\top
      + \left( \vb{Y}^\top \vb{Y} \right)^{-1} \left( \vb{J}^{ji} \right)^\top \\
    &= - \left( \vb{Y}^\top \vb{Y} \right)^{-1}
      \left( \pdv{\vb{Y}^\top}{y_{ji}} \vb{Y} + \vb{Y}^\top \pdv{\vb{Y}}{y_{ji}} \right)
      \left( \vb{Y}^\top \vb{Y} \right)^{-1} \vb{Y}^\top
      + \left( \vb{Y}^\top \vb{Y} \right)^{-1} \left( \vb{J}^{ji} \right)^\top \\
    &= - \left( \vb{Y}^\top \vb{Y} \right)^{-1}
      \left( \left( \vb{J}^{ji} \right)^\top \vb{Y} + \vb{Y}^\top \vb{J}^{ji} \right)
      \left( \vb{Y}^\top \vb{Y} \right)^{-1} \vb{Y}^\top
      + \left( \vb{Y}^\top \vb{Y} \right)^{-1} \left( \vb{J}^{ji} \right)^\top \\
    &= - \left( \vb{Y}^\top \vb{Y} \right)^{-1} \left( \vb{J}^{ji} \right)^\top
      \left( \vb{Y} \left( \vb{Y}^\top \vb{Y} \right)^{-1} \vb{Y}^\top - \vb{I} \right) \\
      & \quad - \left( \vb{Y}^\top \vb{Y} \right)^{-1} \vb{Y}^\top \vb{J}^{ji}
      \left( \vb{Y}^\top \vb{Y} \right)^{-1} \vb{Y}^\top \\
    &= - \left( \vb{Y}^\top \vb{Y} \right)^{-1} \left( \vb{J}^{ji} \right)^\top
      \left( \vb{Y} \vb{Y}^\dagger - \vb{I} \right)
      - \vb{Y}^\dagger \vb{J}^{ji} \vb{Y}^\dagger \\
    &= - \left( \vb{Y}^\top \vb{Y} \right)^{-1} \left( \vb{J}^{ji} \right)^\top
      \left( \vb{Y} \vb{X} - \vb{I} \right) - \vb{X} \vb{J}^{ji} \vb{X}
\end{align*}
$\vb{J}^{ji}$は, $(j, i)$成分のみが$1$, それ以外の成分が$0$の行列.
$\vb{Y} = \vb{X}^\dagger = \left( \vb{X}^\top \vb{X} \right)^{-1} \vb{X}^\top$より,
$\vb{Y} \vb{X} - \vb{I} = \vb{0}$であるから, 結局,
\begin{align*}
  \pdv{\vb{X}}{z_{ji}} = \left( \pdv{\vb{X}}{\vb{X}^\dagger} \right)_{ij}
    = \left( \pdv{\vb{Y}^\dagger}{\vb{Y}} \right)_{ij}
    = \pdv{\left( \vb{Y}^\top \vb{Y} \right)^{-1} \vb{Y}^\top}{y_{ji}}
    = -\vb{X} \vb{J}^{ji} \vb{X}
\end{align*}
\newpage

よって, トレースの定義と, $\left( \vb{J}^{ji} \right)_{kl} = \delta_{jk} \delta_{il}$を使うと,
\begin{align*}
  & \left( \pdv{\ln(\det(\vb{X}^\top \vb{X}))}{\vb{X}^\dagger} \right)_{ij}
    = 2 \tr(\left( \vb{X}^\top \vb{X} \right)^{-1} \vb{X}^\top \pdv{\vb{X}}{z_{ji}}) \\
    &= -2 \tr(\left( \vb{X}^\top \vb{X} \right)^{-1} \vb{X}^\top \vb{X} \vb{J}^{ji} \vb{X})
    = -2 \tr(\vb{J}^{ji} \vb{X}) \\
    &= -2 \sum_k \left( \vb{J}^{ji} \vb{X} \right)_{kk}
    = -2 \sum_k \sum_l \left( \vb{J}^{ji} \right)_{kl} x_{lk} \\
    &= -2 \sum_k \sum_l \delta_{jk} \delta_{il} x_{lk}
    = -2 x_{ij}
\end{align*}

以上で, 行列式を含む微分を確認できました. 長かったですね.
\end{frame}

\section{スカラのスカラによる微分}
\subsection{ベクトルを含む場合}

\subsection{行列を含む場合}

\end{document}

% 誤差逆伝播と絡めた説明
% 二次の微分 (ヘッセ行列)
