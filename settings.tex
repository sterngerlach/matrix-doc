
% settings.tex

\AtBeginSection[]{\frame[t]{\frametitle{目次}
  \tableofcontents[currentsection,hideallsubsections]}}

\AtBeginSubsection[]{\frame[t]{\frametitle{目次}
  \tableofcontents[currentsection,sectionstyle=show/hide,
  currentsubsection,subsectionstyle=show/shaded/hide]}}

\usefonttheme{professionalfonts}
\usetheme{Madrid}

\setbeamercovered{transparent=30} 
% \setbeamertemplate{navigation symbols}{}
\setbeamertemplate{frametitle}[default][left]
\setbeamertemplate{frametitle continuation}{}
\setbeamertemplate{enumerate items}[square]
\setbeamertemplate{caption}[numbered]

\let\oldframe\frame
\renewcommand\frame[1][t,allowdisplaybreaks,allowframebreaks]{\oldframe[#1]}

\addtobeamertemplate{block begin}{\setlength{\abovedisplayskip}{2.5pt}}

\usepackage{bxdpx-beamer}
\usepackage{pxjahyper}
\usepackage{minijs}

\usepackage{amsmath}
\usepackage{amssymb}
\usepackage{amsthm}
\usepackage{bm}
\usepackage{physics}

% Set the path to the figure
\graphicspath{{fig/}}

\usepackage{multirow}

% Add space in the table
\usepackage{cellspace}

% Add space in the table
\setlength\cellspacetoplimit{5pt}
\setlength\cellspacebottomlimit{5pt}

\usepackage{url}

% \hypersetup{
%   colorlinks = true,
%   urlcolor = blue,
%   linkcolor = black,
%   citecolor = green
% }

\DeclareMathOperator*{\argmax}{arg\,max}
\DeclareMathOperator*{\argmin}{arg\,min}
% \DeclareMathOperator{\Tr}{Tr}
% \DeclareMathOperator{\KL}{KL}
\DeclareMathOperator{\diag}{diag}
\DeclareMathOperator{\sgn}{sgn}
\DeclareMathOperator{\adj}{adj}
\DeclareMathOperator{\EOp}{\mathbb{E}}
\DeclareMathOperator{\HOp}{H}
\DeclareMathOperator{\KLOp}{KL}
\newcommand\E[1]{\EOp \left[ #1 \right]}
\newcommand\Entropy[1]{\HOp \left[ #1 \right]}
\newcommand\MutualInfo[1]{I \left( #1 \right)}
\newcommand\KL[2]{\KLOp \left( #1 \parallel #2 \right)}

\usepackage[T1]{fontenc}
\usepackage[utf8]{inputenc}

\setbeamertemplate{theorems}[numbered]
\theoremstyle{definition}
\newtheorem{theorem}{定理}
\newtheorem{definition}{定義}
\newtheorem{proposition}{命題}
\newtheorem{lemma}{補題}
\newtheorem{corollary}{系}
\newtheorem{conjecture}{予想}
\newtheorem*{remark}{Remark}
\renewcommand{\proofname}{}

\renewcommand{\figurename}{図}
\renewcommand{\tablename}{表}

\renewcommand{\kanjifamilydefault}{\gtdefault}

